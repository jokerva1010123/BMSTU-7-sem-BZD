\documentclass[a4paper,14pt]{article}
\usepackage{blindtext}
\usepackage[T2A]{fontenc}
\usepackage[utf8]{inputenc}
\usepackage[english,russian]{babel}
\usepackage{listings}
\usepackage{geometry}
\usepackage{amssymb}
\usepackage{amsmath}
\usepackage[14pt]{extsizes}
\geometry{left=3cm}
\geometry{right=1.5cm}
\geometry{top=2cm}
\geometry{bottom=2cm}
\pagestyle{plain}
\usepackage{pgfplots}
\usepackage{filecontents}
\usepackage{graphicx}
\usepackage{indentfirst}
\DeclareGraphicsExtensions{.png}
\graphicspath{{images/}}
\usetikzlibrary{datavisualization}
\usetikzlibrary{datavisualization.formats.functions}
\usepackage{tabularx}
\pgfplotsset{width=7 cm}
\usepackage{xcolor}
%\renewcommand{\rmdefault}{ftm}
%\usepackage{mathptmx}
\usepackage{setspace}
%\usepackage{minted}
%\полуторный интервал
\onehalfspacing
\frenchspacing

\usepackage{tocloft}
\frenchspacing
\usepackage{multirow}
\usepackage{float}
\usepackage{multirow}

\renewcommand{\cftsecdotsep}{\cftdot}
\renewcommand{\cftsecleader}{\cftdotfill{\cftsecdotsep}}
\renewcommand{\cftsubsecleader}{\cftdotfill{\cftsecdotsep}}
\renewcommand{\cftsubsubsecleader}{\cftdotfill{\cftsecdotsep}}

%\renewcommand\cftchapdotsep{\cftdot}
%\renewcommand\cftsecdotsep{\cftdot}
%\renewcommand{\cftchapleader}{\cftdotfill{\cftchapdotsep}}

% Для измененных титулов глав:
% % подключаем нужные пакеты
%\definecolor{gray75}{gray}{0.75} % определяем цвет
%\newcommand{\hsp}{\hspace{20pt}} % длина линии в 20pt
% titleformat определяет стиль
%\titleformat{\chapter}[hang]{\Huge\bfseries}{\thechapter\hsp\textcolor{black}{|}\hsp}{0pt}{\Huge\bfseries}
%\usepackage{titlesec, blindtext, color}
%\titleformat{\chapter}[hang]{\Huge\bfseries}{\thechapter\hsp\textcolor{black}{|}\hsp}{0pt}{\Huge\bfseries}

% Для листинга кода:
\lstset{ %
extendedchars=\true,
inputencoding=utf8,
morekeywords={include, printf},
texcl=\true,
breaklines=\true,
escapeend=\end{russian},
escapechar=\%,
keepspaces=\true,
language=c,                 % выбор языка для подсветки
basicstyle=\small\sffamily, % размер и начертание шрифта для подсветки кода
numbers=left,               % где поставить нумерацию строк (слева\справа)
numberstyle=\tiny,           % размер шрифта для номеров строк
stepnumber=1,                   % размер шага между двумя номерами строк
numbersep=5pt,                % как далеко отстоят номера строк от подсвечиваемого кода
showspaces=\true,            % показывать или нет пробелы специальными отступами
showstringspaces=\true,      % показывать или нет пробелы в строках
showtabs=false,             % показывать или нет табуляцию в строках
frame=single,              % рисовать рамку вокруг кода
tabsize=4,                 % размер табуляции по умолчанию равен 2 пробелам
captionpos=t,              % позиция заголовка вверху [t] или внизу [b]
breaklines=true,           % автоматически переносить строки (да\нет)
breakatwhitespace=false, % переносить строки только если есть пробел
escapeinside={\//*}{*)}   % если нужно добавить комментарии в коде
}

\begin{document}

\begin{center}
	Рубежный контроль № 3  \\
	по курсу БЖД \\
	Овчинникова А. П., ИУ7-75Б \\
	Билет № 15
\end{center}


\textbf{Задание 1}

\textbf{Основные понятия предмета чрезвычайных ситуаций (ЧС, источники ЧС).}

Чрезвычайная ситуация (ЧС) - это обстановка на определенной территории, сложившаяся под воздействием источника чрезвычайной ситуации, которая может повлечь (или повлекла) за собой человеческие жертвы, ущерб здоровью людей или окружающей природной среде, значительные материальные потери и нарушение условий жизнедеятельности людей.

Источник чрезвычайной ситуации - опасное явление природного, техногенного, биолого-социального или военного характера, в результате которого произошла или может возникнуть ЧС.

\textbf{Задание 2}

\textbf{ВУВ при взрыве конденсированных ВВ.}

Взрывы большинства конденсированных веществ протекают в режиме детонации. Условно все пространство вокруг места взрыва можно разделить на три зоны: зону детонации, зону действия продуктов детонации и зону действия ударной волны.

При взрыве ВУВ распространяется внутри вещества с очень большой скоростью. Из-за малого времени процесса детонации ($\sim 10^{-5}$ с) продукты взрыва не успевают разлететься и образуют зону детонации, представляющую собой облако газа сферической формы с высокой температурой $2000-4000^{\circ}$ К и давлением до 10 ГПа (100 000 кгс/см.кв). Размеры этого облака (зоны) составляют несколько характерных размеров заряда и не зависят от его формы или от вида и состояния окружающей среды.

В зоне за пределами этого облака поражающее действие взрыва определяется действием расширяющихся продуктов детонации и по-прежнему настолько велико, что вызывает безусловно тяжелые последствия. При взрыве на открытом воздухе радиус зоны действия продуктов детонации относительно невелик и составляет около 15 средних радиусов заряда. Если же взрыв происходит в ограниченном пространстве (например в тоннеле), форма этой зоны видоизменяется и ее размеры могу достигать значительной величины, а расширяющиеся газы усиливают метательное действие взрыва, что особенно заметно при взрывах зарядов относительно малой мощности (например 1 кг тротила).

На больших расстояниях от места взрыва на параметры среды продукты детонации уже не оказывают влияния и их значения определяются действием ударной волны и ее затуханием в зависимости от расстояния до места взрыва. Именно эта зона - зона действия ударной волны представляет практический интерес с точки зрения анализа влияния взрыва на степень разрушения зданий сооружений, технику и людей.

Поскольку скорость детонации очень велика, а масса воздуха, вовлекаемая в движение ударной волной, намного превосходит массу заряда, в ходе этого анализа для взрывов на открытом воздухе можно условно принять следующие допущения: при взрыве конденсированного ВВ энергия выделяется в точке; на всем расстоянии от точки взрыва до точки анализа его последствий действует одна и таже зависимость между параметрами ударной волны и удалением от места взрыва.

\textbf{Задание 3}

\textbf{ В 15 м от узкой стены кирпичного промышленного здания на бетоне взорвалось 90 кг тротила. Определить зоны возможных разрушений в здании длиной 50м.}

Определим тротиловый эквивалент взрыва:

$M_{T} = 2 \eta k M_{BB} = 2 \cdot 0.95 \cdot 1 \cdot 90 = 171$ кг.

Определим зоны возможных разрушений.

\textbf{1. Зона полных разрушений -- $\Delta P_{\phi} = 120$ кПа.}

Приведенный радиус взрыва:

$\overline{R} = \sqrt[3]{\left[ 1 + \frac{337}{\Delta P_{\phi}} \right]^2 - 1} = 2.38$

Радиус зоны:

$R = \overline{R} \cdot \sqrt[3]{M_{T}} = 2.38 \cdot \sqrt[3]{171} = 13.21$ м.

Длина полных разрушений в здании:

$L = R - r_0 = 13.21 - 15 = -1.79$ м.

$L < 0$, поэтому этой зоны разрушений в здании нет.

\newpage
\textbf{2. Зона сильных разрушений -- $\Delta P_{\phi} = 85$ кПа.}


Приведенный радиус взрыва:

$\overline{R} = \sqrt[3]{\left[ 1 + \frac{337}{\Delta P_{\phi}} \right]^2 - 1} = 2.87$

Радиус зоны:

$R = \overline{R} \cdot \sqrt[3]{M_{T}} = 2.87 \cdot \sqrt[3]{171} = 15.93$ м.

Длина полных разрушений в здании:

$L = R - r_0 = 15.93 - 15 = 0.93$ м.

$L > 0$, поэтому эта зона разрушений в здании есть.

\textbf{3. Зона средних разрушений -- $\Delta P_{\phi} = 50$ кПа.}

Приведенный радиус взрыва:

$\overline{R} = \sqrt[3]{\left[ 1 + \frac{337}{\Delta P_{\phi}} \right]^2 - 1} = 3.89$

Радиус зоны:

$R = \overline{R} \cdot \sqrt[3]{M_{T}} = 3.89 \cdot \sqrt[3]{171} = 21.59$ м.

Длина полных разрушений в здании:

$L = R - r_0 = 21.59 - 15 = 6.59$ м.

$L > 0$, поэтому эта зона разрушений в здании есть.

\textbf{4. Зона слабых разрушений -- $\Delta P_{\phi} = 30$ кПа.}

Приведенный радиус взрыва:

$\overline{R} = \sqrt[3]{\left[ 1 + \frac{337}{\Delta P_{\phi}} \right]^2 - 1} = 5.3$

Радиус зоны:

$R = \overline{R} \cdot \sqrt[3]{M_{T}} = 5.3 \cdot \sqrt[3]{171} = 29.42$ м.

Длина полных разрушений в здании:

$L = R - r_0 = 29.42 - 15 = 14.42$ м.

$L > 0$, поэтому эта зона разрушений в здании есть.


\textbf{5. Зона слабых остекления -- $\Delta P_{\phi} = 12$ кПа.}

Приведенный радиус взрыва:

$\overline{R} = \sqrt[3]{\left[ 1 + \frac{337}{\Delta P_{\phi}} \right]^2 - 1} = 9.45$

Радиус зоны:

$R = \overline{R} \cdot \sqrt[3]{M_{T}} = 9.45 \cdot \sqrt[3]{171} = 52.45$ м.

Длина полных разрушений в здании:

$L = R - r_0 = 52.45 - 15 = 37.45$ м.

$L > 0$, поэтому эта зона разрушений в здании есть.

\end{document}
